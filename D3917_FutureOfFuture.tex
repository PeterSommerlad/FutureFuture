\documentclass[ebook,11pt,article]{memoir}
\usepackage{geometry}                % See geometry.pdf to learn the layout options. There are lots.
\geometry{a4paper}                   % ... or a4paper or a5paper or ... 
%\geometry{landscape}                % Activate for for rotated page geometry
%\usepackage[parfill]{parskip}    % Activate to begin paragraphs with an empty line rather than an indent

\usepackage[final]
           {listings}     % code listings
\usepackage{color}        % define colors for strikeouts and underlines
\usepackage{underscore}   % remove special status of '_' in ordinary text
\usepackage{xspace}
\pagestyle{myheadings}
\markboth{D3917 2014-02-13}{D3917 2014-02-13}
%%TODO
% namespace std { inline namespace literals { inline namespace chrono_literals, string_literals, complex_literals
% put b on the side
% complex: if, i, il optional i_l

\title{On the Future of std::future and a Future concept and Data-flow Programming}
\author{Hartmut Kaiser, Thomas Heller, Peter Sommerlad}
\date{2014-02-13}                                           % Activate to display a given date or no date
\input{macros}
\setsecnumdepth{subsection}

\begin{document}
\maketitle
\begin{tabular}[t]{|l|l|}\hline 
Document Number: & D3917 \\\hline
Date: & 2014-02-13 \\\hline
Project: & Programming Language C++\\\hline 
\end{tabular}
\chapter{History}
The implementation of std::future mixes the concept of a future and the concept of asynchronous operation. This has lead to several arguments on the mailing list and other discussions on possible extensions or limitations of std::future and its surrounding standard library infrastructure.

\section{Discussion on c++std-parallel mailing list}

In 2013 there have been several discussions raised by papers (N3857,N3784, etc) \textbf{(find numbers)} that asked for extending \tcode{std::future} API with a member function \tcode{future::then()} that allows to specify a function that will run after the future object becomes ready, relying on the asynchronousity of the underlying operation making the future's shared state ready. The invocation of .then() would then return a future wrapping the original future object, etc. 

Peter strongly objected to the abstraction of future gain "fat" by giving it more than the semantic of a \emph{"ticket for a value or exception to be obtained later"}. While a concrete implementation such as std::future in the world of C++11 requires some hooking to a synchronization mechanism, the abstraction should be agnostic about where the value it eventually receives comes from. His colleague Luc Bläser also supports his opinion and provides examples how chaining of tasks could be achieved with existing C++11 mechanisms.

To quote Luc's concerns:

\begin{quote}
I also tend to share your standpoint, namely that futures are probably not the most appropriate concept for incorporating continuations. Futures and tasks are generally two different concepts: a future represents a value of possibly future computation, while a tasks abstracts a unit of asynchronous or parallel execution. Therefore, as you also stated, the future can but does not necessarily have to represent the result of a parallel or asynchronous task.

Chaining continuations on futures may thus lead to the following consequences:
\begin{itemize}
\item Conceptual confusion: Futures would need to define an underlying task model, i.e. a synchronization point and thread pool mechanism. The notion of future would thus be subtly altered to the abstraction of a task, i.e. deviate from the common terminology.
\item Implementation restriction: The two fairly separate concerns, namely result access and task chaining, are coupled in one interface, thus limiting degrees of freedom for implementation. 

\end{itemize}

It would be probably more flexible if proper full-fledged tasks could be chained (each task having a future), or, if asynchronous operations are defined to be triggered on some available futures: i.e. something like async(future1, future2, [] (result1, result2) { continuation }). The latter is naturally already supported without syntactic extensions, i.e. async([]{ x = future1.get(); y = future2.get(); continuation(x, y); }). A benefit of this could be that the chaining of a single or multiple preceding tasks is uniformly supported. Multiplexed selection of "any" first available futures could be possibly enabled by providing an any-combinator of futures. future3 = any(future1, future2). 
\end{quote}

Others seem to have the perspective that a \tcode{std::future} actually is about synchronization and thus chaining execution of code with respect to the event of a \tcode{std::future} instance becoming ready is the way to provide an attractive style of "continuation-based" programming \textbf{(check terminology)}.

In addition std::future doesn't fit the needs for any task-based parallelism, because its tight coupling to std::promise (and std::thread) and its heavy-weight. 

We conclude that there is a need to clarify underlying concepts, before we continue to fight about adding stuff to the library and where. 



\chapter{Introduction}
All the above issues led to the discovery by Olivier Giroux while discussing with Peter that there is a need for a concept \emph{Future} for representing the "later value" that is more light-weight than std::future and a separate concept for the synchronization aspect of std::future that comes close to a "task completion" that might happen to also be supported by std::future or not.

\section{Attempt to show how future::then could be done without changing future}

\section{Attempt to specify the core of a concept Future}
Ticket vs. blocking/sync

\section{Channel vs. Future-Promise}
Refer to channel. Kohlhoff's paper N3896

alternative views on the requested mechanism of composeable asynchronous operations (async/future as is of C++14 do not compose).


What goes in here? get(), what else?
\section{Open Issues to be Discussed}

\section{Acknowledgements}
Acknowledgements go to Luc Bläser, Olivier Giroux, many more.
%more

\chapter{Proposed Library Additions}
%%%%%%%%%%% C++ EDITOR %%%%% 
%
%Add the following declaration in [tuple.general] in the synopsis under the group \emph{element access}:
%
%\begin{codeblock}
%template <typename F, typename Tuple>
%decltype(auto) apply(F&& f, Tuple&& t);
%\end{codeblock}

%%%%%%%%%%% C++ EDITOR %%%%% 



%%%%
%%%%%

\end{document}   